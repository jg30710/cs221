\documentclass[12pt]{article}
\usepackage{fullpage,enumitem,amsmath,amssymb,graphicx}
\title{Homework 7}

\begin{document}

\begin{center}
{\Large CS221 Fall 2014 Homework 7}

\begin{tabular}{rl}
SUNet ID: & thomasme \\
Name: & Thomas Mendoza \\
Collaborators: & Steven Samson
\end{tabular}
\end{center}

By turning in this assignment, I agree by the Stanford honor code and declare
that all of this is my own work.

\section*{Problem 1}

\begin{enumerate}[label=(\alph*)]
	\item The posterior distribution can be expressed with the following formula:

		\[
			p(c_1 = 1 | d_1 = 0) = p(c_1 = 1) \cdot p(d_1 = 0 | c_1 = 1)
		\]

		The probability distribution contained in \(p(c_1 = 1)\) is dependent
		on the value of \(c_0\). The distribution for the posterior is
		represented by the following table:

		\begin{tabular}{| l | l |}
			\hline
			\(c_0 = 0, c_1 = 1, d_1 = 0\) & \(0.5 \epsilon \eta\)\\ \hline
			\(c_0 = 1, c_1 = 1, d_1 = 0\) & \(0.5 (1-\epsilon) \eta\)\\ \hline
		\end{tabular}

	\item Given that we now know that \(d_2 = 1\), the posterior becomes:

		\[
			p(c_1 = 1 | d_1 = 0, d_2 = 1) = p(c_1 = 1)
				\cdot p(d_1 = 0, d_2 = 1 | c_1 = 1)
		\]

		Much akin to the previous case, we now compute the distribution, only
		this time introducing one more unknown, so to speak, in the form
		of the value of \(c_2\) since \(d_2\) depends on it:

		\begin{tabular}{| l | l |}
			\hline
			\(c_0 = 0, c_1 = 1, d_1 = 0, c_2 = 0\) &
				\(0.5 \epsilon \eta \epsilon\)\\ \hline
			\(c_0 = 1, c_1 = 1, d_1 = 0, c_2 = 0\) &
				\(0.5 (1-\epsilon) \eta \epsilon\)\\ \hline
			\(c_0 = 0, c_1 = 1, d_1 = 0, c_2 = 1\) &
				\(0.5 \epsilon \eta (1-\epsilon)\)\\ \hline
			\(c_0 = 1, c_1 = 1, d_1 = 0, c_2 = 1\) &
				\(0.5 (1-\epsilon) \eta (1-\epsilon)\)\\ \hline
		\end{tabular}

	\item Finally, knowing that \(\epsilon = 0.1\) and \(\eta = 0.2\) using the 
		previous tables, we can compute the posterior distribution. The
		tables are now:

		\begin{tabular}{| l | l | l |}
			\hline
			\(c_0 = 0, c_1 = 1, d_1 = 0\) & \(0.5 \epsilon \eta\) & 0.01\\ \hline
			\(c_0 = 1, c_1 = 1, d_1 = 0\) & \(0.5 (1-\epsilon) \eta\) & 0.09\\ \hline
		\end{tabular}

		\begin{tabular}{| l | l | l |}
			\hline
			\(c_0 = 0, c_1 = 1, d_1 = 0, c_2 = 0\) &
				\(0.5 \epsilon \eta \epsilon\) & 0.001\\ \hline
			\(c_0 = 1, c_1 = 1, d_1 = 0, c_2 = 0\) &
			   \(0.5 (1-\epsilon) \eta \epsilon\) & 0.009\\ \hline
			\(c_0 = 0, c_1 = 1, d_1 = 0, c_2 = 1\) &
			   \(0.5 \epsilon \eta (1-\epsilon)\) & 0.009\\ \hline
			\(c_0 = 1, c_1 = 1, d_1 = 0, c_2 = 1\) &
			   \(0.5 (1-\epsilon) \eta (1-\epsilon)\) & 0.081\\ \hline
		\end{tabular}

		Totaling the values yields both the posteriors being \(0.1\).
		My intuition was that, by knowing future sensor readings, the
		strange \(c_1 = 1\) and \(d_1 = 0\) readings would be reconciled
		because the next reading of \(d_2 = 1\) would reaffirm that
		the car was previously at \(c_1\) despite the reading at \(d_1\).

\end{enumerate}

\section*{Problem 5}

\begin{enumerate}[label=(\alph*)]
	\item Stuff
\end{enumerate}

\end{document}
