\documentclass[12pt]{article}
\usepackage{fullpage,enumitem,amsmath,amssymb,graphicx}
\title{Homework 8}

\begin{document}

\begin{center}
{\Large CS221 Fall 2014 Homework 8}

\begin{tabular}{rl}
SUNet ID: & thomasme \\
Name: & Thomas Mendoza \\
Collaborators: & Steven Samson
\end{tabular}
\end{center}

By turning in this assignment, I agree by the Stanford honor code and declare
that all of this is my own work.

\section*{Problem 4}

\begin{enumerate}[label=(\alph*)]
	\item First to convert the KB to CNF:
		\[
			KB = \{(A \lor B) \to C, A\}
		\]
		\[
			KB = \{\lnot(A \lor B) \lor C, A\}
		\]
		\[
			KB = \{(\lnot A \lor \lnot B) \lor C, A\}
		\]
		\[
			KB = \{(\lnot A \lor C) \land (\lnot B \lor C), A\}
		\]
		Converting items of the form \(\lnot P \lor Q\) to \(P \to Q\):
		\[
			KB = \{(A \to C) \land (B \to C), A\}
		\]
		Using the fact that conjuction in the KB can be expressed as separate
		entities in the KB:
		\[
			KB = \{(A \to C), (B \to C), A\}
		\]
		Running modus ponens, knowing \(A\) and \(A \to C\) we are able to derive
		\(C\) and the algorithm converges.

	\item Reduce the KB to CNF
		\[
			KB = \{A \lor B, B \to C, (A \lor C) \to D\}
		\]
		\[
			KB = \{A \lor B, \lnot B \lor C, \lnot (A \lor C) \lor D\}
		\]
		\[
			KB = \{A \lor B, \lnot B \lor C, (\lnot A \land \lnot C) \lor D\}
		\]
		\[
			KB = \{A \lor B, \lnot B \lor C, (\lnot A \lor D) \land (\lnot C \lor D)\}
		\]
		\[
			KB = \{A \lor B, \lnot B \lor C, (\lnot A \lor D), (\lnot C \lor D)\}
		\]

		Now we apply resolution until we converge to \(D\):
		\[
			KB = \{A \lor C, (\lnot A \lor D), (\lnot C \lor D)\}
		\]
		\[
			KB = \{A \lor D, (\lnot A \lor D)\}
		\]
		\[
			KB = \{D\}
		\]


\end{enumerate}

\section*{Problem 5}

\begin{enumerate}[label=(\alph*)]
	\item The reason that adding the constraint "A number is not larger than itself" has no finite
		model that will be consistent is that in order to test its validity, all values, \(x\), must
		be considered and that process maps \(x\) to the natural numbers (which are countably infinite
		and not able to be represented by a finite model).

		In other words, testing a subset of the natural numbers is insufficient for asserting validity
		of the constraint and must therefore return false unless all natural numbers have been considered.
\end{enumerate}

\end{document}
