\documentclass[12pt]{article}
\usepackage{fullpage,enumitem,amsmath,amssymb,graphicx}

\begin{document}

\begin{center}
{\Large CS221 Fall 2014 Homework 2}

\begin{tabular}{rl}
SUNet ID: & thomasme \\
Name: & Thomas Mendoza \\
Collaborators: & Steven Samson
\end{tabular}
\end{center}

By turning in this assignment, I agree by the Stanford honor code and declare
that all of this is my own work.

	\section*{Problem 1}

	\begin{enumerate}[label=(\alph*)]
		\item To compute weights using stochastic gradient descent, we first
			define each feature vector for the reviews ordered
			{'pretty', 'good', 'bad', 'plot', 'scenery'}:

			\begin{gather*}
				\Phi(x_1) = [1, 1, 0, 0, 0, 0] \text{ and } y = 1 \\
				\Phi(x_2) = [0, 0, 1, 1, 0, 0] \text{ and } y = -1 \\
				\Phi(x_3) = [0, 0, 1, 0, 1, 0] \text{ and } y = 1 \\
				\Phi(x_4) = [1, 0, 0, 0, 0, 1] \text{ and } y = 1
			\end{gather*}

			Now the computation of
			\(w \leftarrow \eta \nabla_w \text{Loss}_\text{hinge}(x,y,w)\)
			can proceed.
			We initialize \(w\) to the vector of all zeros and begin computation
			over the set \(\Phi(x)\) vectors assuming \(\eta = 1\). First will
			be the computation of the gradient of the hinge loss followed by
			the update of \(w\):
			\[
				w \cdot \Phi(x_1) = [0, 0, 0, 0, 0, 0] \cdot [1, 1, 0, 0, 0, 0] = 0
			\]
			since the margin is less than \(1\) we return \(-\Phi(x_1)y\) and update
			\(w\) appropriately:
			\[
				w = [0, 0, 0, 0, 0, 0] + [1, 1, 0, 0, 0, 0] = [1, 1, 0, 0, 0, 0]
			\]
			we continue in the same fashion for the remaining \(\Phi(x)\).
			For \(\Phi(x_2)\):
			\[
				w \cdot \Phi(x_2) = [1, 1, 0, 0, 0, 0] \cdot [0, 0, 1, 1, 0, 0] = 0
			\]
			since the margin is less than \(1\) we return \(-\Phi(x_2)y\) and update
			\(w\):
			\[
				w = [1, 1, 0, 0, 0, 0] - [0, 0, 1, 1, 0, 0] = [1, 1, -1, -1, 0, 0]
			\]
			For \(\Phi(x_3)\):
			\[
				w \cdot \Phi(x_3) = [1, 1, -1, -1, 0, 0] \cdot [0, 0, 1, 0, 1, 0] = -1
			\]
			since the margin is less than \(1\) we return \(-\Phi(x_3)y\) and update
			\(w\):
			\[
				w = [1, 1, -1, -1, 0, 0] + [0, 0, 1, 0, 1, 0] = [1, 1, 0, -1, 1, 0]
			\]
			For \(\Phi(x_4)\):
			\[
				w \cdot \Phi(x_4) = [1, 1, 0, -1, 1, 0] \cdot [1, 0, 0, 0, 0, 1] = 1
			\]
			since the margin is equal to \(1\) I choose to return
			\(-\Phi(x_4)y\) and update \(w\):
			\[
				w = [1, 1, 0, -1, 1, 0] + [1, 0, 0, 0, 0, 1] = [2, 1, 0, -1, 1, 1]
			\]
			making the final value for \(w = [2, 1, 0, -1, 1, 1]\).

		\item Creating a set of vectors from the set of words {'not', 'good', 'bad'}
			we have the vectors:
			\begin{gather*}
				\Phi(x_1) = [0, 1, 0] \text{ and } y = 1 \\
				\Phi(x_2) = [0, 0, 1] \text{ and } y = -1 \\
				\Phi(x_3) = [1, 1, 0] \text{ and } y = -1 \\
				\Phi(x_4) = [1, 0, 1] \text{ and } y = 1
			\end{gather*}
			Although the set of \(\Phi(x)\) consists of four vectors, only 3 are
			unique, \(\Phi(x_3)\) for instance can be written as:
			\[
				\Phi(x_3) = \Phi(x_1) - \Phi(x_2) + \Phi(x_4)
			\]
			meaning that, at most, \(w\) can properly fit \(3\) of the \(4\)
			vectors. This is because the output values \(y\) form a four
			dimensional vector, but the set of current \(\Phi(x)\) only
			map to three dimensions. The way that we can solve this is by 
			adding a new feature that makes these vectors linearly independent.
			The feature I propose is word connotation product
			(ex: not * bad = \(-1 \cdot -1 = 1\)).
			This transforms the vectors to the following:
			\begin{gather*}
				\Phi(x_1) = [0, 1, 0, 0] \text{ and } y = 1 \\
				\Phi(x_2) = [0, 0, 1, 0] \text{ and } y = -1 \\
				\Phi(x_3) = [1, 1, 0, -1] \text{ and } y = -1 \\
				\Phi(x_4) = [1, 0, 1, 1] \text{ and } y = 1
			\end{gather*}
			These vectors are linearly independent and have a \(w\)
			for which there is \(0\) error. That one such solution
			is the vector:
			\[
				[1, 1, -1, 3]
			\]

		\item The generaliztion of the hinge loss with movie review
			rankings amounts to the following
			\[
				\text{Loss}_\text{hinge} =
					\begin{cases}
						1 - \sum_{j=0}^k (w \cdot \Phi(x_i) - max(w \cdot \Phi(x\notin I))) \text{ where } i \in I_j & \text{ if margin less than 0 } \\
						0 & \text{ margin greater than 1 } 
					\end{cases}
			\]
			The max in this piecewise function can be precomputed since it will be the same for each item in the sum.

	\end{enumerate}

	\section*{Problem 2}

	\begin{enumerate}[label=(\alph*)]
		\item Programming assignment 
		\item Programming assignment 
		\item Programming assignment 

		\item I chose \(\eta = 0.1\) and numIters to be 5. The reasoning for this
			is that this was the combination that reduced overfitting amongst other
			test cases. Increasing iterations would keep the error down, but the 
			train error would be depressed more significantly as well.
			The associated training and test errors are as follows:
			\begin{gather*}
				\eta = 0.1 \text{ and numIters }= 5:
					\text{Official: train error = 0.0751266178953, dev error = 0.294316263365} \\
				\eta = 0.5 \text{ and numIters }= 5:
					\text{Official: train error = 0.0925717501407, dev error = 0.300787844682} \\
				\eta = 0.5 \text{ and numIters }= 10:
					\text{Official: train error = 0.0241980866629, dev error = 0.281091727631} \\
				\eta = 1 \text{ and numIters }= 10:
					\text{Official: train error = 0.0250422059651, dev error = 0.286156443444} \\
				\eta = 1 \text{ and numIters }= 5:
					\text{Official: train error = 0.164884637029, dev error = 0.330050647158}
			\end{gather*}

		\item Analyzing the wrong answers from the output:

			=== seldom has a movie so closely matched the spirit of a man and his work .
			Truth: 1, Prediction: -1 [WRONG]

				Here, there are an abundance of stop-words that were weighted heavily negative.

			=== a heady , biting , be-bop ride through nighttime manhattan , a loquacious videologue of the modern male and the lengths to which he'll go to weave a protective cocoon around his own ego .
			Truth: 1, Prediction: -1 [WRONG]

				Again, several negative stop words, but also a few positive words that had 0 weight. If the advanced vocabulary were better weighted the review would have been correct.

			=== wickedly funny , visually engrossing , never boring , this movie challenges us to think about the ways we consume pop culture .
			Truth: 1, Prediction: -1 [WRONG]

				This one contains several stop-words, heavily weighted negative words (boring) and negative punction weights.

			=== rain is a small treasure , enveloping the viewer in a literal and spiritual torpor that is anything but cathartic .
			Truth: 1, Prediction: -1 [WRONG]

				This sentence contains very few words with positive weight.

			=== only in its final surprising shots does rabbit-proof fence find the authority it's looking for .
			Truth: -1, Prediction: 1 [WRONG]

				Problematic because it contained words that usually have negative connotation to positively review the movie.

			=== even during the climactic hourlong cricket match , boredom never takes hold .
			Truth: 1, Prediction: -1 [WRONG]

				More aggravated problem than the previous example, it uses several negative words to create a positive review.

			=== time out is as serious as a pink slip . and more than that , it's an observant , unfussily poetic meditation about identity and alienation .
			Truth: 1, Prediction: -1 [WRONG]

				Again, too many negative stop-words.

			=== alternative medicine obviously has its merits . . . but ayurveda does the field no favors .
			Truth: -1, Prediction: 1 [WRONG]

				This review contained the use of ellipses and periods have negative weight.

			=== no screen fantasy-adventure in recent memory has the showmanship of clones' last 45 minutes .
			Truth: 1, Prediction: -1 [WRONG]

				The hyphenated fantasy adventure portion was categorized as a single element and could have contributed positively.

			=== a clever blend of fact and fiction .
			Truth: 1, Prediction: -1 [WRONG]

				This review was short and the few key words carried negative weight.

		\item Programming assignment

		\item Running learnPredictor using extractCharacterFeatures had training and dev
			nearly on par with extractWordFeatures as \(n\) grew larger, \(n = 4\) in fact
			had nearly identical error. The reason for this is that the n-grams capture
			an additional piece of information--word proximity. This can be thought of
			in the manner that reviews of similar rating (positive or negative) use
			similar phrases (hence the words are often collocated) which is captured by the list of n-grams
			since they are not limited to returning data from a single word.
			A sample sentence that would be better analyzed by n-grams would be:

			The movie was not bad.

			Not and bad would most likely be made heavily negative after training with
			extractWordFeatures since those words (in any location in the sentence)
			appear more often in bad reviews, but n-grams would catch the proximity
			of not and bad and be able to find that occurrence as a common phrase across
			several good reviews.

	\end{enumerate}

	\section*{Problem 3}

	\begin{enumerate}[label=(\alph*)]
		\item Given points:
			\begin{gather*}
				\Phi(x_1) = [0, 1] \\
				\Phi(x_2) = [2, 1] \\
				\Phi(x_3) = [0, 0] \\
				\Phi(x_4) = [0, 2]
			\end{gather*}
			we run k-means for centroids
			\(\mu_1 = [0, -1] \text{ and } \mu_2 = [2, 2]\)
			by first assigning points to the centroids:
			\begin{gather*}
				\text{min} \{ ( [1,0] - [0, -1] )^2, ( [1,0] - [2, 2] )^2\} \\
				\text{min} \{ ( [1,1] )^2, ( [-1, -2] )^2\} \\
				\text{min} \{ 4, 5\} \\
				\Phi(x_1) \rightarrow \mu_1
			\end{gather*}
			\begin{gather*}
				\text{min} \{ ( [2,1] - [0, -1] )^2, ( [2,1] - [2, 2] )^2\} \\
				\text{min} \{ ( [2,2] )^2, ( [0, -1] )^2\} \\
				\text{min} \{ 8, 1\} \\
				\Phi(x_2) \rightarrow \mu_2
			\end{gather*}
			\begin{gather*}
				\text{min} \{ ( [0,0] - [0, -1] )^2, ( [0,0] - [2, 2] )^2\} \\
				\text{min} \{ ( [0,-1] )^2, ( [-2, -2] )^2\} \\
				\text{min} \{ 1, 8\} \\
				\Phi(x_3) \rightarrow \mu_1
			\end{gather*}
			\begin{gather*}
				\text{min} \{ ( [0,2] - [0, -1] )^2, ( [0,2] - [2, 2] )^2\} \\
				\text{min} \{ ( [0,3] )^2, ( [-2, 0] )^2\} \\
				\text{min} \{ 9, 4\} \\
				\Phi(x_4) \rightarrow \mu_2
			\end{gather*}
			Then we recompute the centroids from the point vector averages:
			\begin{gather*}
				\mu_1 = \frac{1}{2} ( [1,0] + [0, 0] ) =  [0.5, 0] \\
				\mu_2 = \frac{1}{2} ( [2,1] + [0, 2] ) =  [1, 1.5] \\
			\end{gather*}
			Now we see if any of the clusters change:
			\begin{gather*}
				\text{min} \{ ( [1,0] - [0.5, 0] )^2, ( [1,0] - [1, 1.5] )^2\} \\
				\text{min} \{ ( [0.5,0] )^2, ( [0, -1.5] )^2\} \\
				\text{min} \{ 0.25, 2.25\} \\
				\Phi(x_1) \rightarrow \mu_1
			\end{gather*}
			\begin{gather*}
				\text{min} \{ ( [2,1] - [0.5, 0] )^2, ( [2,1] - [1, 1.5] )^2\} \\
				\text{min} \{ ( [1.5,1] )^2, ( [1, -0.5] )^2\} \\
				\text{min} \{3.25, 1.25\} \\
				\Phi(x_2) \rightarrow \mu_2
			\end{gather*}
			\begin{gather*}
				\text{min} \{ ( [0,0] - [0.5, 0] )^2, ( [0,0] - [1, 1.5] )^2\} \\
				\text{min} \{ ( [-0.5,0] )^2, ( [-1, -1.5] )^2\} \\
				\text{min} \{ 0.25, 3.25\} \\
				\Phi(x_3) \rightarrow \mu_1
			\end{gather*}
			\begin{gather*}
				\text{min} \{ ( [0,2] - [0.5, 0] )^2, ( [0,2] - [1, 1.5] )^2\} \\
				\text{min} \{ ( [0.5,2] )^2, ( [-1, 0.5] )^2\} \\
				\text{min} \{ 4.25, 1.25\} \\
				\Phi(x_4) \rightarrow \mu_2
			\end{gather*}
			The assignments have stopped changing and so we've reached end result (1):
			\[
				\mu_1 = [0.5, 0] \text{ and } \mu_2 = [1, 1.5]
			\]
			\begin{gather*}
				\Phi(x_1) \rightarrow \mu_1 \\
				\Phi(x_2) \rightarrow \mu_2 \\
				\Phi(x_3) \rightarrow \mu_1 \\
				\Phi(x_4) \rightarrow \mu_2
			\end{gather*}
			Now for the next set of centroids
			\(\mu_1 = [2, 0] \text{ and } \mu_2 = [-1, 0]\)
			\begin{gather*}
				\text{min} \{ ( [1,0] - [2, 0] )^2, ( [1,0] - [-1, 0] )^2\} \\
				\text{min} \{ ( [-1,0] )^2, ( [2, 0] )^2\} \\
				\text{min} \{ 1, 4\} \\
				\Phi(x_1) \rightarrow \mu_1
			\end{gather*}
			\begin{gather*}
				\text{min} \{ ( [2,1] - [2, 0] )^2, ( [2,1] - [-1, 0] )^2\} \\
				\text{min} \{ ( [0,1] )^2, ( [3, 1] )^2\} \\
				\text{min} \{ 1, 10\} \\
				\Phi(x_2) \rightarrow \mu_1
			\end{gather*}
			\begin{gather*}
				\text{min} \{ ( [0,0] - [2, 0] )^2, ( [0,0] - [-1, 0] )^2\} \\
				\text{min} \{ ( [-2,0] )^2, ( [1, 0] )^2\} \\
				\text{min} \{ 1, 8\} \\
				\Phi(x_3) \rightarrow \mu_1
			\end{gather*}
			\begin{gather*}
				\text{min} \{ ( [0,2] - [2, 0] )^2, ( [0,2] - [-1, 0] )^2\} \\
				\text{min} \{ ( [-2,2] )^2, ( [1, 2] )^2\} \\
				\text{min} \{ 8, 5\} \\
				\Phi(x_4) \rightarrow \mu_2
			\end{gather*}
			Then we recompute the centroids from the point vector averages:
			\begin{gather*}
				\mu_1 = \frac{1}{3} ( [1,0] + [2, 1] + [0, 0] ) =  [1, 0.3] \\
				\mu_2 = [0, 2] \\
			\end{gather*}
			Check the assignments again...
			\begin{gather*}
				\text{min} \{ ( [1,0] - [1, 0.3] )^2, ( [1,0] - [0, 2] )^2\} \\
				\text{min} \{ ( [0,0.3] )^2, ( [1, 2] )^2\} \\
				\text{min} \{ 0.9, 5\} \\
				\Phi(x_1) \rightarrow \mu_1
			\end{gather*}
			\begin{gather*}
				\text{min} \{ ( [2,1] - [1, 0.3] )^2, ( [2,1] - [0, 2] )^2\} \\
				\text{min} \{ ( [1,0.7] )^2, ( [2, -1] )^2\} \\
				\text{min} \{ 1.49, 5\} \\
				\Phi(x_2) \rightarrow \mu_1
			\end{gather*}
			\begin{gather*}
				\text{min} \{ ( [0,0] - [1, 0.3] )^2, ( [0,0] - [0, 2] )^2\} \\
				\text{min} \{ ( [-1,-0.3] )^2, ( [0, 2] )^2\} \\
				\text{min} \{ 1.9, 4\} \\
				\Phi(x_3) \rightarrow \mu_1
			\end{gather*}
			\begin{gather*}
				\text{min} \{ ( [0,2] - [1, 0.3] )^2, ( [0,2] - [0, 2] )^2\} \\
				\text{min} \{ ( [-1,1.7] )^2, ( [0, 0] )^2\} \\
				\text{min} \{ 2.69, 0\} \\
				\Phi(x_4) \rightarrow \mu_2
			\end{gather*}
			The assignments have stopped changing and so we've reached end result (2):
			\[
				\mu_1 = [1, 0.3] \text{ and } \mu_2 = [0, 2]
			\]
			\begin{gather*}
				\Phi(x_1) \rightarrow \mu_1 \\
				\Phi(x_2) \rightarrow \mu_1 \\
				\Phi(x_3) \rightarrow \mu_1 \\
				\Phi(x_4) \rightarrow \mu_2
			\end{gather*}


		\item Programming assignment
		\item Knowing that certain points belong in a particular cluster, the k-means
			algorithm could be represented by the following pseudo code:

			\begin{verbatim}
				# Preprocess the points belonging to a particular cluster by
				# finding the centroid between them
				def bestCentroidForClusters:
					for k clusters:
						for e pinnedExamples:
							# compute centroid, mark centroid as pinned

				# Carry on computation as usual, skipping centroid assignment and movement
				# for the pinned points
				def assignPointsToCentroids:
					for e examples:
						if examples[e] is not pinned:
						# business as usual, carry on
				def bestCentroidForClusters:
					for k clusters:
						for e examples:
							# compute centroid
							if k is not pinned:
								# Carry on computation as previously done
				for i in range(maxIters):
					assignPointsToCentroids()
					bestCentroidForClusters()
			\end{verbatim}
	\end{enumerate}

\end{document}
